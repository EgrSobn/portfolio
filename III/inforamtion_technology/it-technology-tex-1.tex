% "Лабораторная работа"
 
\documentclass[a4paper,12pt]{article} % тип документа
% report, book
%  Русский язык 
\usepackage[T2A]{fontenc}			% кодировка
\usepackage[utf8]{inputenc}			% кодировка исходного текста
\usepackage[english,russian]{babel}	% локализация и переносы

% Математика
\usepackage{amsmath,amsfonts,amssymb,amsthm,mathtools} 
 
\usepackage{wasysym}

%Заговолок
\author{Егор Собинин, 1 группа 2 подгруппа}
\title{Основы работы в \LaTeX{}}
\date{\today}

\begin{document} % начало документа

\maketitle
\newpage

%Название документа: Основы работы в LaTeX
%Автор документа: ФИ, группа
%Дата
%Раздел 1. Издательские системы
%Подраздел 1. Издательская система TeX
%Подраздел 2. Дональд Кнут
%Подраздел 3. Издательская система LaTeX
%Подраздел 4. Лесли Лэмпорт
%Раздел 2. Основные правила создания текстового документа.

\section{Издательские системы}
\hspace{15mm}Настольная издательская система (НИС) — комплект оборудования для подготовки оригинал-макета издания, готового для передачи в типографию.
Как правило, в состав НИС включают одну или несколько персональных рабочий станций с программным обеспечением для создания макета оформления, распознания, набора и вёрстки текста, редактирования изображений, предпечатной подготовки оригинал-макета. В состав НИС могут также входить принтер (для вывода промежуточных результатов и плёнок) и сканер.
\subsection{Издательская система TeX}
\hspace{15mm}TeX — система компьютерной вёрстки, разработанная американским профессором информатики Дональдом Кнутом в целях создания компьютерной типографии. В неё входят средства для секционирования документов, для работы с перекрёстными ссылками. В частности, благодаря этим возможностям, TeX популярен в академических кругах, особенно среди математиков и физиков. Название произносится как «тех».
\subsection{Дональд Кнут}
\hspace{15mm}Дональд Эрвин Кнут -- американский учёный в области информатики.
Доктор философии (1963), эмерит-профессор Стэнфордского университета, член Американского философского общества (2012), преподаватель и идеолог программирования, автор 19 монографий (в том числе ряда классических книг по программированию) и более 160 статей, разработчик нескольких известных программных технологий. Автор всемирно известной серии книг, посвящённой основным алгоритмам и методам вычислительной математики, а также создатель настольных издательских систем TeX и METAFONT, предназначенных для набора и вёрстки книг научно-технической тематики (в первую очередь — физико-математических).
\subsection{Издательская система LaTeX}
\hspace{15mm}\LaTeX  — наиболее популярный набор макрорасширений (или макропакет) системы компьютерной вёрстки TeX, который облегчает набор сложных документов. В типографском наборе системы TeX форматируется традиционно как \LaTeX.

\hspace{15mm}Важно заметить, что ни один из макропакетов для TeX’а не может расширить возможностей TeX (всё, что можно сделать в LaTeX’е, можно сделать и в TeX’е без расширений), но, благодаря различным упрощениям, использование макропакетов зачастую позволяет избежать весьма изощрённого программирования.

\hspace{15mm}Пакет позволяет автоматизировать многие задачи набора текста и подготовки статей, включая набор текста на нескольких языках, нумерацию разделов и формул, перекрёстные ссылки, размещение иллюстраций и таблиц на странице, ведение библиографии и др. Кроме базового набора существует множество пакетов расширения LaTeX. Первая версия была выпущена Лесли Лэмпортом в 1984 году; текущая версия, LaTeX2, после создания в 1994 году испытывала некоторый период нестабильности, окончившийся к концу 2000-х годов, а в настоящее время стабилизировалась (хотя раз в год выходит новая версия).
\subsection{Лесли Лэмпорт}
\hspace{15mm}Лесли Лэмпорт (англ. Leslie Lamport; род. 7 февраля 1941, Нью-Йорк, Нью-Йорк) — американский учёный в области информатики, первый лауреат премии Дейкстры. Разработчик LaTeX — популярного набора макрорасширений системы компьютерной вёрстки TeX, исследователь теории распределённых систем, темпоральной логики и вопросов синхронизации процессов во взаимодействующих системах. Лауреат Премии Тьюринга 2013 года.
\section{Основные правила создания текстового документа.}
\hspace{15mm}Между словами ставится только один пробел.
\begin{itemize}
\itemПереход на новую строку в процессе набора текста происходит автоматически, не требуя ввода специального символа.
\itemОкончание абзаца маркируется нажатием клавиши Enter, позволяющей перейти на новую строку — первую строку нового абзаца.
\itemПеред знаками препинания (такими, как ;:.,!?) пробел не ставится. Перед тире вводится пробел. После любого знака препинания вводится один пробел или символ конца абзаца.
\itemЗнак «дефис» в словах вводится без пробелов.
\itemПосле открывающих и перед закрывающими скобками ({}()[]) и кавычками пробел не вводится.
\itemДля ввода римских цифр используются прописные латинские буквы I, V, X, L, С, D, М.
\itemЗнак «неразрывный (нерастяжимый) пробел» препятствует символам, между которыми он поставлен, располагаться на разных строчках, и сохраняется фиксированным при любом выравнивании абзаца (не может увеличиваться, в отличие от обычного пробела). Этот знак очень удобно применять при вводе дат (которые не принято располагать на двух строчках), фамилий с инициалами и т. п. Например: А. С. Пушкин. Ставится знак «неразрывный пробел» с помощью одновременного нажатия комбинации клавиш Ctrl + Shift + пробел.
\end{itemize}
\end {document}