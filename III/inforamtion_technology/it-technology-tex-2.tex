% "Лабораторная работа"
 
\documentclass[a4paper,12pt]{article} % тип документа
% report, book
%  Русский язык 
\usepackage[T2A]{fontenc}			% кодировка
\usepackage[utf8]{inputenc}			% кодировка исходного текста
\usepackage[english,russian]{babel}	% локализация и переносы

% Математика
\usepackage{amsmath,amsfonts,amssymb,amsthm,mathtools} 
\usepackage{wasysym}

%Заговолок
\author{Егор Собинин, 1 группа 2 подгруппа}
\title{Инвариантная самостоятельная работа 5. Команды создания текста.}
\date{\today}

\begin{document} % начало документа
\maketitle
\newpage
\section{Основные команды}
\begin{enumerate} 
\item \verb|\documentclass| - описывает класс документа, статья, книга, отчет и так далее;
\item \verb|\begin|- указывает на начало тела документа или блока;
\item \verb|\end| - завершение документа или блока;
\item \verb|\usepackage| - загружает пакет команд LaTeX в текущий документ, нужно для настройки кодировки, шрифта и другого;
\item \verb|\maketitle| - создает титульный лист с названием и всем прочим;
\item \verb|\tableofcontents| - содержание статьи или книги;
\item \verb|\chapter| - создает главу;
\item \verb|\section| - создает раздел;
\item \verb|\subsection| - создает подраздел;
\item \verb|\bfseries| - жирный текст;
\item \verb|\textit| - курсив;
\item \verb|\title| - заголовок документа;
\item \verb|\author| - автор документа;
\item \verb|\date| - дата создания документа.
\end{enumerate}
\section{Основы форматирования}
\hspace{10mm}Каждый документ LaTeX имеет определенную структуру, вначале идут настройки отображения, имортирование нужных пакетов, а уже потом сам текст в теле документа. Вот эти строки инициализируют основные параметры:
\begin{enumerate}
\item \verb|\documentclass[a4paper,11pt]{book}|
\item \verb|\usepackage{amsmath,amsthm,amssymb}|
\item \verb|\usepackage[T1,T2A]{fontenc}|
\item \verb|\usepackage[utf8]{inputenc}|
\item \verb|\usepackage[english,russian]{babel}|
\end{enumerate}
\end {document}