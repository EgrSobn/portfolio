% "Лабораторная работа"
 
\documentclass[a4paper,12pt]{article} % тип документа
% report, book
%  Русский язык 
\usepackage[T2A]{fontenc}			% кодировка
\usepackage[utf8]{inputenc}			% кодировка исходного текста
\usepackage[english,russian]{babel}	% локализация и переносы

% Математика
\usepackage{amsmath,amsfonts,amssymb,amsthm,mathtools} 
\usepackage{wasysym}

%Заговолок
\author{Егор Собинин, 1 группа 2 подгруппа}
\title{Лабораторная работа 6. Особености технологии набора технического текста}
\date{\today}

\begin{document} % начало документа
\maketitle
\newpage
\section{Для чего предназначена издательская система LaTeX?}

\hspace{5mm}\LaTeX - наиболее популярный набор макрорасширений (или макропакет) системы компьютерной вёрстки TeX, который облегчает набор сложных документов. \textbf{В типографском наборе} системы TeX форматируется традиционно как \LaTeX.

\begin{flushleft}Важно заметить, что ни один из макропакетов для \textbf{TeX’а} не может расширить возможностей TeX (всё, что можно сделать в LaTeX’е, можно сделать и в TeX’е без расширений), но, благодаря различным упрощениям, использование макропакетов зачастую позволяет избежать весьма изощрённого программирования.
\end{flushleft}
\begin{center}
Пакет позволяет автоматизировать многие задачи набора текста и подготовки статей, включая набор текста на нескольких языках, нумерацию разделов и формул, перекрёстные ссылки, размещение иллюстраций и таблиц на странице, ведение библиографии и др. Кроме базового набора существует множество пакетов расширения \LaTeX.
\end{center}
\section{В каких случаях рационально её использовать?}
\begin{flushright}
Хотя \LaTeX не так неудобен для пользователя, как TeX, написание в нём обычного текста можно считать аномалией. \LaTeX — система набора и вёрстки и язык разметки. Системы набора и вёрстки обычно не используются для редактирования текста, и хотя языки разметки вроде \textbf{XML} и \textbf{HTML} часто используются таким образом, обычно это считается плохой идеей.
\end{flushright}
\section{Какие преимущества имеет работа в этот системе?}
\begin{center}
\begin{itemize}
    \item Готовые профессионально выполненные макеты, делающие документы действительно выглядящими <<как изданные>>.
    \item Удобно поддержана верстка математических формул.
    \item Пользователю нужно выучить лишь несколько понятных команд, задающих логическую структуру документа. Ему практически никогда не нужно возиться собственно с макетом документа.
    \item Легко изготавливаются даже сложные структуры, типа примечаний, оглавлений, библиографий и прочее.
    \item Существуют свободно распространяемые дополнительные пакеты для многих типографских задач, не поддерживаемых напрямую базовым \LaTeX. Например, наличествуют пакеты для включения POSTSCRIPT графики или для верстки библиографий в точном соответствии с конкретными стандартами. Многие из этих дополнительных компонент описаны в.
    \item \LaTeX поощряет авторов писать хорошо структурированные документы, так как именно так \LaTeX и работает -- путем спецификации структуры.
    \item TEX, форматирующее сердце \textit{LATEX2e}, чрезвычайно мобилен и свободно доступен. Поэтому система работает практически на всех существующих платформах.
\end{itemize}
\end{center}
\section{Какие сложности могут возникнуть при работе в этот системе?}
\begin{itemize}
    \item Специфический софт.
    \item Специфический синтаксис.
    \item Отсуттсвие привычного графического интерфейса.
    \item Необходимость <<вручную>> подключать пакеты.
\end{itemize}
\section{Какие недостатки отмечают пользователи при работе с этой системой?}
\begin{itemize}
    \item Трудное обучение верстке документов.
    \item Мало инструкций на русском языке.
    \item Необходимо знание английского языка.
\end{itemize}

\href{http://fkn.ktu10.com/?q=node/2906}{ - Справочник по LaTeX}
\end{document}