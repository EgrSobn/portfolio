% "Лабораторная работа"
 
\documentclass[a4paper,12pt]{article} % тип документа
% report, book
%  Русский язык 
\usepackage[T2A]{fontenc}			% кодировка
\usepackage[utf8]{inputenc}			% кодировка исходного текста
\usepackage[english,russian]{babel}	% локализация и переносы

% Математика
\usepackage{amsmath,amsfonts,amssymb,amsthm,mathtools} 
\usepackage{wasysym}

%Заговолок
\author{Егор Собинин, 1 группа 2 подгруппа}
\title{Инвариантная самостоятельная работа 6}
\date{\today}
\begin{document} % начало документа
\maketitle
\newpage
\section{Вопросы}

\hspace{5mm}\emph{Какая основная цель написания текста?} - передать читателю идеи, информацию или знания.

\emph{Каким способом можно поставить троеточие?} - \verb|\ldots|

\emph{Как сделать разрыв страницы?} - \verb|\newpage|

\emph{Как указать текущую дату на текущем языке} - \verb|\today|

\emph{Как указать знак $\sim$} - \verb|$\sim$|

\emph{Как подключить португальский язык?} - \verb|\usepackage[portuguese]{babel}|

\emph{Как указать заголовок 1 уровня?} - \verb|\section{}|

\emph{Как указать заголовок 2 уровня?} - \verb|\subsection|

\emph{Как сделать титульный лист} - \verb|\maketitle|

\emph{Как сделать сноску} - \verb|\footnote{}|

\section{Таблица}

\begin{tabular}{l||l||cc}
 \hline
 № & Название & Команда\\
 \hline\hline
 1 & разрыв страницы & \verb|\newpage|\\
 2 & знак $\sim$ & \verb|$\sim$|\\
 3 & троеточие & \verb|\ldots|\\
 4 & титульный лист & \verb|\maketitle|\\
 5 & заголовок 1 уровня & \verb|\section{}|\\
 6 & заголовок 2 уровня & \verb|\subsection|\\
 7 & сноска & \verb|\footnote{}|\\
 8 & ссылка & \verb|\url{}|\\
 9 & выделение слова & \verb|\emph{}|\\
 10 & размещение по центру & \verb|\begin{center}\end{center}|
 \end{tabular}

\end{document}
