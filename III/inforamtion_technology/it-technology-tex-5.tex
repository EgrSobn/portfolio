% "Лабораторная работа"
 
\documentclass[a4paper,12pt]{article} % тип документа
% report, book
%  Русский язык 
\usepackage[T2A]{fontenc}			% кодировка
\usepackage[utf8]{inputenc}			% кодировка исходного текста
\usepackage[english,russian]{babel}	% локализация и переносы

% Математика
\usepackage{amsmath,amsfonts,amssymb,amsthm,mathtools} 
\usepackage{wasysym}

%Заговолок
\author{Егор Собинин, 1 группа 2 подгруппа}
\title{Лабораторная работа 7. Особенности технологии создания текста с формулами}
\date{\today}
\begin{document} % начало документа
\maketitle
\newpage
\section{Формулы}
$$ y = ax^2+bx+c$$
$$ y = 2^{n-1}+3$$
$$y = 
 \begin{cases}
   t, \text{при } x\geq3\\
   a \cdot x-s, \text{при } x\in (-5.5 ; 3)\\
   x^3, \text{при } x\leq -5.5
 \end{cases}$$
$$y = t \cdot x^3 + k \cdot x + s$$
$$ y = 2^{k+2}-5$$
$$y = 
 \begin{cases}
   w^2, \text{при } x\geq 1.5\\
   n \cdot x + 9, \text{при } x\in (-12 ; 1.5)\\
   c-x, \text{при } x\leq -12
 \end{cases}$$
 $$\int \frac{dx}{\ln x} = \ln | \ln | + \sum_{i-1}^\infty \frac{(\ln x)^i}{i \cdot i!}$$
 $$\int \frac{dx}{(\ln x)^n} = -\frac{x}{(n-1)(\ln x)^{n-1}} + \frac{1}{n-1} \int \frac{dx}{(\ln x)^{n-1}} \text{ для } n \neq 1$$
 $$\int x^m \ln x dx = x ^ {m+1} (\frac{\ln x}{m+1} - \frac{1}{(m+1)^2}) \text{ для } m \neq -1$$
 $$\int x^m(\ln x)^n dx = \frac{x^{m+1}(\ln x)^n}{m+1} - \frac{n}{m+1} \int x^m(\ln x)^{n-1} \text{ для } m \neq -1$$
 $$\int \frac{(\ln x)^{n}dx}{x} = \frac{(\ln x)^{n+1}}{n+1} \text{ для } n \neq -1$$
 $$\int \frac{\ln x dx}{x^m} = - \frac{\ln x}{(m-1)x^{m-1}} - \frac{1}{(m-1)^2x^{m-1}} \text{ для } m \neq 1$$
\end{document}