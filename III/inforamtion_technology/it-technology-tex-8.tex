% "Лабораторная работа"
 
\documentclass[a4paper,12pt]{article} % тип документа
% report, book
%  Русский язык 
\usepackage[T2A]{fontenc}			% кодировка
\usepackage[utf8]{inputenc}			% кодировка исходного текста
\usepackage[english,russian]{babel}	% локализация и переносы

% Математика
\usepackage{amsmath,amsfonts,amssymb,amsthm,mathtools} 
\usepackage{wasysym}

%Заговолок
\author{Егор Собинин, 1 группа 2 подгруппа}
\title{Лабораторня работа 8.Создание матриц средствами LaTeX}
\date{\today}
\begin{document} % начало документа
\maketitle
\newpage

\section{Умножение матрицы на число}

\textbf{Дано:}
\begin{equation*}
A = \left(
\begin{array}{ccc}
1 & 2 & 3\\
4 & 5 & 6 
\end{array}
\right)
\end{equation*}

\textbf{Решение:}
\begin{equation*}
B = 2 \times A = 2 \times \left(
\begin{array}{ccc}
1 & 2 & 3\\
4 & 5 & 6 
\end{array}
\right) = \left(
\begin{array}{ccc}
2 & 4 & 6\\
8 & 10 & 12 
\end{array}
\right)
\end{equation*}

\section{Умножение матрицы на матрицу}

\textbf{Дано:}
\begin{equation*}
A = \left(
\begin{array}{ccc}
2 & 3 & 1\\
-1 & 0 & 1 
\end{array}
\right)
\end{equation*}
\begin{equation*}
B = \left(
\begin{array}{cc}
2 & 1\\
-1 & 1\\
3 & -2
\end{array}
\right)
\end{equation*}

\textbf{Решение:}
\begin{equation*}
    C = A \times B = A = \left(
\begin{array}{ccc}
2 & 3 & 1\\
-1 & 0 & 1 
\end{array}
\right) \times \left(
\begin{array}{cc}
2 & 1\\
-1 & 1\\
3 & -2
\end{array}
\right) = \left(
\begin{array}{cc}
4 & 3\\
-2 & -1
\end{array}
\right)
\end{equation*}

\section{Транспонирование матрицы}

\textbf{Дано:}
\begin{equation*}
A = \left(
\begin{array}{ccc}
7 & 8 & 9\\
1 & 2 & 3 
\end{array}
\right)
\end{equation*}

\textbf{Решение:}
\begin{equation*}
A = \left(
\begin{array}{ccc}
7 & 8 & 9\\
1 & 2 & 3 
\end{array}
\right) \Rightarrow A^T = \left(
\begin{array}{cc}
7 & 1\\
8 & 2\\
9 & 3
\end{array}
\right)
\end{equation*}

\section{Обратные матрицы}

\textbf{Дано:}
\begin{equation*}
A = \left(
\begin{array}{cc}
2 & -1\\
3 & 1 
\end{array}
\right)
\end{equation*}

\textbf{Решение:}
\begin{equation*}
\det A = A = \left|
\begin{array}{cc}
2 & -1\\
3 & 1 
\end{array}
\right| = 2 \times 1 - 3 \times (-1) = 5 \neq 0
\end{equation*}
\begin{equation*}
A^V = \left(
\begin{array}{cc}
1 & -3\\
1 & 2 
\end{array}
\right)
\end{equation*}
\begin{equation*}
    (A^V)^T = \left(
\begin{array}{cc}
1 & 1\\
-3 & 2 
\end{array}
\right)
\end{equation*}
\begin{equation*}
A^{-1} = \frac{1}{\det A}(A^V)^T=\frac{1}{5} \left(
\begin{array}{cc}
1 & 1\\
-3 & 2 
\end{array}
\right) = \left(
\begin{array}{cc}
\frac{1}{5} & \frac{1}{5}\\
- \frac{3}{5} & \frac{1}{2}
\end{array}
\right)
\end{equation*}

\textbf{Ответ:}
\begin{equation*}
    A^{-1} = \left(
\begin{array}{cc}
\frac{1}{5} & \frac{1}{5}\\
- \frac{3}{5} & \frac{1}{2}
\end{array}
\right)
\end{equation*}
\end{document}