% "Лабораторная работа"
 
\documentclass[a4paper,12pt]{article} % тип документа
% report, book
%  Русский язык 
\usepackage[T2A]{fontenc}      % кодировка
\usepackage[utf8]{inputenc}      % кодировка исходного текста
\usepackage[english,russian]{babel}  % локализация и переносы

% Математика
\usepackage{amsmath,amsfonts,amssymb,amsthm,mathtools} 
 
\usepackage{wasysym}

%Заговолок
\author{Егор Собинин, 1 группа 2 подгруппа}
\title{Вариативное задание 6}
\date{\today}
\begin{document}

\maketitle

\section{Ресурсы интернета, содержащие рекомендации по работе в LaTeX}
\begin{itemize}
    \item 
    \verb*|https://ru.wikibooks.org/wiki/LaTeX/Управление_библиографией| - ссылка на инофрмацию по цитированию
    \item 
    \verb*|https://habr.com/ru/company/ruvds/blog/574352/|- ссылка на Хабр, которая помагает разобраться в азах Latex
    \item
    \verb*|www.youtube.com/watch?v=8dCm1V1XDzw&ab_channel=Диджитализируй%21| - ссылка на основы в видеоформате
    \item 
    \verb*|http://fkn.ktu10.com/?q=node/7430| - ссылка на источник как вставить ссылки
        \item 
        \verb|http://mydebianblog.blogspot.com/2008/12/latex_15.html| - ссылка с информацией как вставлять картинки
        \item
        \verb |https://losst.pro/kak-polzovatsya-latex| - основные команды Latex
        \item
        \verb |https://ru.overleaf.com/project| - ссылка на Latex Online
        \item
        \verb|ru.wikibooks.org/wiki/Математические_формулы_в_LaTeX| - математические формулы в Latex
    
\end{itemize}


\end{document}